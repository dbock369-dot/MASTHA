\documentclass{article}

% Packages
\usepackage[utf8]{inputenc}        % Eingabekodierung
\usepackage[T1]{fontenc}           % Schriftkodierung
\usepackage[ngerman]{babel}        % Sprache (für Silbentrennung etc.)
\usepackage[a4paper, left=3cm, right=2.5cm, top=2.5cm, bottom=2.5cm]{geometry}
\usepackage{setspace}
\onehalfspacing  
\usepackage{lmodern}               % Moderne serifenbetonte Schrift (Standard)
\usepackage{graphicx}             % Für Bilder
\usepackage{siunitx}
\usepackage{amsmath, amssymb}     % Mathematisches
\usepackage{hyperref}             % Klickbare Links im PDF
\usepackage{caption}              % Bild-/Tabellenunterschriften besser steuern
\usepackage{csquotes}             % Für saubere Zitate
\usepackage{fancyhdr}

\pagestyle{fancy}
\fancyhf{} 
\fancyhead[L]{\nouppercase{\leftmark}}
\fancyhead[R]{\thepage}

\begin{document}

% Formatierung aufgehoben für Titelseite
\begin{titlepage}
\thispagestyle{empty} % keine Kopf-/Fußzeile auf dieser Seite
\fancypagestyle{plain}{%
  \fancyhf{} % alle Felder leeren
  \renewcommand{\headrulewidth}{0pt} % keine Linie oben
  \renewcommand{\footrulewidth}{0pt} % keine Linie unten
}
\pagestyle{plain} % wendet obige Definition an

	\vspace*{-2.5cm}\noindent%
	\hspace*{-0.40cm}\noindent%
	\includegraphics[width=0.50\linewidth]{abb/unilogo_de.pdf}
 \hspace*{2.85cm}\noindent
	

	
	\vskip 2.0 cm
	\begin{center}
		 %%Bachelorarbeit Nr. 130 \\
		\vskip 1.5 cm
		{\Huge\bfseries Nutzung von Bayescher Optimierung und DFT zur bestimmung von optimaler Nanoporösen Strukturen im Kontext der Adsorption \\
			\par}
            \vskip 1 cm
   \text{Masterarbeit am ITT der Universität Stuttgart}
		\vskip 1 cm
	%	 Praktikumsdauer \\
    %XX. Monat XXXX bis XX. Monat XXXX \\
		 %\vskip 2.0 cm
		 %Oktober 2024
		 %\vskip 1.5 cm
		% Universität Stuttgart \\
		% Fakultät für Energie-, Verfahrens- und Biotechnik \\
		% Institut für Chemische Verfahrenstechnik
		% \vskip 2.0 cm

		 
		 
		 \renewcommand{\arraystretch}{1.4}  
		\begin{tabular}{ll}
                Verfasser: \\ 
                Daniel Bock (3383944) \\
               \\

			Betreuerin: Dr.-Ing. Gernot Bauer  \\
			Prüfer: Prof. Dr.-Ing. Joachim Groß

		\end{tabular}
	

  \vskip 1.5 cm
  
		 Fakultät 7: Konstruktions-, Produktions-, und Fahrzeugtechnik \\
   Studiengang Maschinenbau (Master) \\ der Universität Stuttgart
		
		\vskip 1.5 cm

                        \textbf{\today}
  \end{center}

\newpage




\end{titlepage}

\input{Aufgabenstellung}
\input{Bearbeitungen/a)}
\input{Bearbeitungen/b)}
\input{Bearbeitungen/c)}
\input{Bearbeitungen/d)}

\end{document}